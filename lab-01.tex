% Options for packages loaded elsewhere
\PassOptionsToPackage{unicode}{hyperref}
\PassOptionsToPackage{hyphens}{url}
%
\documentclass[
]{article}
\usepackage{amsmath,amssymb}
\usepackage{iftex}
\ifPDFTeX
  \usepackage[T1]{fontenc}
  \usepackage[utf8]{inputenc}
  \usepackage{textcomp} % provide euro and other symbols
\else % if luatex or xetex
  \usepackage{unicode-math} % this also loads fontspec
  \defaultfontfeatures{Scale=MatchLowercase}
  \defaultfontfeatures[\rmfamily]{Ligatures=TeX,Scale=1}
\fi
\usepackage{lmodern}
\ifPDFTeX\else
  % xetex/luatex font selection
\fi
% Use upquote if available, for straight quotes in verbatim environments
\IfFileExists{upquote.sty}{\usepackage{upquote}}{}
\IfFileExists{microtype.sty}{% use microtype if available
  \usepackage[]{microtype}
  \UseMicrotypeSet[protrusion]{basicmath} % disable protrusion for tt fonts
}{}
\makeatletter
\@ifundefined{KOMAClassName}{% if non-KOMA class
  \IfFileExists{parskip.sty}{%
    \usepackage{parskip}
  }{% else
    \setlength{\parindent}{0pt}
    \setlength{\parskip}{6pt plus 2pt minus 1pt}}
}{% if KOMA class
  \KOMAoptions{parskip=half}}
\makeatother
\usepackage{xcolor}
\usepackage[margin=1in]{geometry}
\usepackage{color}
\usepackage{fancyvrb}
\newcommand{\VerbBar}{|}
\newcommand{\VERB}{\Verb[commandchars=\\\{\}]}
\DefineVerbatimEnvironment{Highlighting}{Verbatim}{commandchars=\\\{\}}
% Add ',fontsize=\small' for more characters per line
\usepackage{framed}
\definecolor{shadecolor}{RGB}{248,248,248}
\newenvironment{Shaded}{\begin{snugshade}}{\end{snugshade}}
\newcommand{\AlertTok}[1]{\textcolor[rgb]{0.94,0.16,0.16}{#1}}
\newcommand{\AnnotationTok}[1]{\textcolor[rgb]{0.56,0.35,0.01}{\textbf{\textit{#1}}}}
\newcommand{\AttributeTok}[1]{\textcolor[rgb]{0.13,0.29,0.53}{#1}}
\newcommand{\BaseNTok}[1]{\textcolor[rgb]{0.00,0.00,0.81}{#1}}
\newcommand{\BuiltInTok}[1]{#1}
\newcommand{\CharTok}[1]{\textcolor[rgb]{0.31,0.60,0.02}{#1}}
\newcommand{\CommentTok}[1]{\textcolor[rgb]{0.56,0.35,0.01}{\textit{#1}}}
\newcommand{\CommentVarTok}[1]{\textcolor[rgb]{0.56,0.35,0.01}{\textbf{\textit{#1}}}}
\newcommand{\ConstantTok}[1]{\textcolor[rgb]{0.56,0.35,0.01}{#1}}
\newcommand{\ControlFlowTok}[1]{\textcolor[rgb]{0.13,0.29,0.53}{\textbf{#1}}}
\newcommand{\DataTypeTok}[1]{\textcolor[rgb]{0.13,0.29,0.53}{#1}}
\newcommand{\DecValTok}[1]{\textcolor[rgb]{0.00,0.00,0.81}{#1}}
\newcommand{\DocumentationTok}[1]{\textcolor[rgb]{0.56,0.35,0.01}{\textbf{\textit{#1}}}}
\newcommand{\ErrorTok}[1]{\textcolor[rgb]{0.64,0.00,0.00}{\textbf{#1}}}
\newcommand{\ExtensionTok}[1]{#1}
\newcommand{\FloatTok}[1]{\textcolor[rgb]{0.00,0.00,0.81}{#1}}
\newcommand{\FunctionTok}[1]{\textcolor[rgb]{0.13,0.29,0.53}{\textbf{#1}}}
\newcommand{\ImportTok}[1]{#1}
\newcommand{\InformationTok}[1]{\textcolor[rgb]{0.56,0.35,0.01}{\textbf{\textit{#1}}}}
\newcommand{\KeywordTok}[1]{\textcolor[rgb]{0.13,0.29,0.53}{\textbf{#1}}}
\newcommand{\NormalTok}[1]{#1}
\newcommand{\OperatorTok}[1]{\textcolor[rgb]{0.81,0.36,0.00}{\textbf{#1}}}
\newcommand{\OtherTok}[1]{\textcolor[rgb]{0.56,0.35,0.01}{#1}}
\newcommand{\PreprocessorTok}[1]{\textcolor[rgb]{0.56,0.35,0.01}{\textit{#1}}}
\newcommand{\RegionMarkerTok}[1]{#1}
\newcommand{\SpecialCharTok}[1]{\textcolor[rgb]{0.81,0.36,0.00}{\textbf{#1}}}
\newcommand{\SpecialStringTok}[1]{\textcolor[rgb]{0.31,0.60,0.02}{#1}}
\newcommand{\StringTok}[1]{\textcolor[rgb]{0.31,0.60,0.02}{#1}}
\newcommand{\VariableTok}[1]{\textcolor[rgb]{0.00,0.00,0.00}{#1}}
\newcommand{\VerbatimStringTok}[1]{\textcolor[rgb]{0.31,0.60,0.02}{#1}}
\newcommand{\WarningTok}[1]{\textcolor[rgb]{0.56,0.35,0.01}{\textbf{\textit{#1}}}}
\usepackage{graphicx}
\makeatletter
\def\maxwidth{\ifdim\Gin@nat@width>\linewidth\linewidth\else\Gin@nat@width\fi}
\def\maxheight{\ifdim\Gin@nat@height>\textheight\textheight\else\Gin@nat@height\fi}
\makeatother
% Scale images if necessary, so that they will not overflow the page
% margins by default, and it is still possible to overwrite the defaults
% using explicit options in \includegraphics[width, height, ...]{}
\setkeys{Gin}{width=\maxwidth,height=\maxheight,keepaspectratio}
% Set default figure placement to htbp
\makeatletter
\def\fps@figure{htbp}
\makeatother
\setlength{\emergencystretch}{3em} % prevent overfull lines
\providecommand{\tightlist}{%
  \setlength{\itemsep}{0pt}\setlength{\parskip}{0pt}}
\setcounter{secnumdepth}{-\maxdimen} % remove section numbering
\ifLuaTeX
  \usepackage{selnolig}  % disable illegal ligatures
\fi
\usepackage{bookmark}
\IfFileExists{xurl.sty}{\usepackage{xurl}}{} % add URL line breaks if available
\urlstyle{same}
\hypersetup{
  pdftitle={Lab 01 - UK Attractions},
  pdfauthor={Aniruddh Iyer},
  hidelinks,
  pdfcreator={LaTeX via pandoc}}

\title{Lab 01 - UK Attractions}
\author{Aniruddh Iyer}
\date{2024-10-02}

\begin{document}
\maketitle

\subsubsection{Load packages and data}\label{load-packages-and-data}

\begin{Shaded}
\begin{Highlighting}[]
\CommentTok{\# Do not edit this code chunk!}
\FunctionTok{library}\NormalTok{(tidyverse) }
\end{Highlighting}
\end{Shaded}

\begin{Shaded}
\begin{Highlighting}[]
\CommentTok{\# Do not edit this code chunk!}
\NormalTok{visitors }\OtherTok{\textless{}{-}} \FunctionTok{read\_csv}\NormalTok{(}\StringTok{"data/UK{-}visitor{-}numbers.csv"}\NormalTok{)}
\end{Highlighting}
\end{Shaded}

\section{Wrangling Data}\label{wrangling-data}

\subsection{Question 1}\label{question-1}

\emph{How many tourist attractions are there in the data set?}

\begin{Shaded}
\begin{Highlighting}[]
\NormalTok{visitors }\SpecialCharTok{\%\textgreater{}\%} \FunctionTok{count}\NormalTok{()}
\end{Highlighting}
\end{Shaded}

\begin{verbatim}
## # A tibble: 1 x 1
##       n
##   <int>
## 1   348
\end{verbatim}

\subsubsection{Exercise a.}\label{exercise-a.}

\emph{Create a frequency table of the number of tourist attractions in
the data set by \texttt{region}.}

\begin{Shaded}
\begin{Highlighting}[]
\CommentTok{\# Delete the comment \textquotesingle{}\#\textquotesingle{} symbol at the start of the next line and complete the code. }

\NormalTok{visitors }\SpecialCharTok{\%\textgreater{}\%} \FunctionTok{count}\NormalTok{(admission)}
\end{Highlighting}
\end{Shaded}

\begin{verbatim}
## # A tibble: 3 x 2
##   admission     n
##   <chr>     <int>
## 1 Charged      38
## 2 Free        128
## 3 Members     182
\end{verbatim}

\subsubsection{Exercise b.}\label{exercise-b.}

\emph{Create a frequency table by \texttt{admission} and
\texttt{setting}.}

\begin{Shaded}
\begin{Highlighting}[]
\CommentTok{\# Delete the comment \textquotesingle{}\#\textquotesingle{} symbol at the start of the next line and complete the code.}

\NormalTok{visitors }\SpecialCharTok{|\textgreater{}} \FunctionTok{count}\NormalTok{(region)}
\end{Highlighting}
\end{Shaded}

\begin{verbatim}
## # A tibble: 12 x 2
##    region                       n
##    <chr>                    <int>
##  1 East Midlands                9
##  2 East of England             26
##  3 London                      39
##  4 North East                  25
##  5 North West                  21
##  6 Northern Ireland             9
##  7 Scotland                    97
##  8 South East                  34
##  9 South West                  30
## 10 Wales                        2
## 11 West Midlands               35
## 12 Yorkshire and the Humber    21
\end{verbatim}

\begin{Shaded}
\begin{Highlighting}[]
\NormalTok{visitors }\SpecialCharTok{\%\textgreater{}\%} \FunctionTok{count}\NormalTok{(admission, setting)}
\end{Highlighting}
\end{Shaded}

\begin{verbatim}
## # A tibble: 9 x 3
##   admission setting     n
##   <chr>     <chr>   <int>
## 1 Charged   I          21
## 2 Charged   M          12
## 3 Charged   O           5
## 4 Free      I          77
## 5 Free      M           8
## 6 Free      O          43
## 7 Members   I          23
## 8 Members   M         106
## 9 Members   O          53
\end{verbatim}

\subsection{Question 2}\label{question-2}

\emph{What are the variable data types?}

\begin{Shaded}
\begin{Highlighting}[]
\FunctionTok{class}\NormalTok{(visitors}\SpecialCharTok{$}\NormalTok{n\_2022)}
\end{Highlighting}
\end{Shaded}

\begin{verbatim}
## [1] "numeric"
\end{verbatim}

\begin{Shaded}
\begin{Highlighting}[]
\NormalTok{visitors }\SpecialCharTok{\%\textgreater{}\%} \FunctionTok{summarise\_all}\NormalTok{(class)}
\end{Highlighting}
\end{Shaded}

\begin{verbatim}
## # A tibble: 1 x 6
##   attraction n_2021  n_2022  admission setting   region   
##   <chr>      <chr>   <chr>   <chr>     <chr>     <chr>    
## 1 character  numeric numeric character character character
\end{verbatim}

\subsection{Question 3}\label{question-3}

\emph{Which attraction had the most number of visitors in 2022?}

\begin{Shaded}
\begin{Highlighting}[]
\NormalTok{visitors }\SpecialCharTok{\%\textgreater{}\%} \FunctionTok{arrange}\NormalTok{(}\FunctionTok{desc}\NormalTok{(n\_2022))}
\end{Highlighting}
\end{Shaded}

\begin{verbatim}
## # A tibble: 348 x 6
##    attraction                             n_2021 n_2022 admission setting region
##    <chr>                                   <dbl>  <dbl> <chr>     <chr>   <chr> 
##  1 The Crown Estate, Windsor Great Park   5.40e6 5.64e6 Free      O       South~
##  2 Natural History Museum (South Kensing~ 1.57e6 4.65e6 Free      I       London
##  3 The British Museum                     1.33e6 4.10e6 Free      I       London
##  4 Tate Modern                            1.16e6 3.88e6 Free      I       London
##  5 Southbank Centre                       7.87e5 2.95e6 Free      I       London
##  6 The National Gallery                   7.09e5 2.73e6 Free      I       London
##  7 V&A South Kensington                   8.58e5 2.37e6 Free      I       London
##  8 Somerset House                         9.85e5 2.35e6 Free      M       London
##  9 Science Museum                         9.56e5 2.33e6 Free      I       London
## 10 Tower of London                        5.26e5 2.02e6 Members   M       London
## # i 338 more rows
\end{verbatim}

\subsubsection{Exercise c.}\label{exercise-c.}

\emph{What are the top 10 most visited attractions in 2021?}

\begin{Shaded}
\begin{Highlighting}[]
\NormalTok{visitors }\SpecialCharTok{\%\textgreater{}\%} 
  \FunctionTok{arrange}\NormalTok{(}\FunctionTok{desc}\NormalTok{(n\_2021)) }\SpecialCharTok{\%\textgreater{}\%}
  \FunctionTok{head}\NormalTok{(}\AttributeTok{n =} \DecValTok{10}\NormalTok{)}
\end{Highlighting}
\end{Shaded}

\begin{verbatim}
## # A tibble: 10 x 6
##    attraction                             n_2021 n_2022 admission setting region
##    <chr>                                   <dbl>  <dbl> <chr>     <chr>   <chr> 
##  1 The Crown Estate, Windsor Great Park   5.40e6 5.64e6 Free      O       South~
##  2 Royal Botanic Gardens Kew              1.96e6 1.96e6 Members   M       London
##  3 Natural History Museum (South Kensing~ 1.57e6 4.65e6 Free      I       London
##  4 RHS Garden Wisley                      1.41e6 1.49e6 Members   O       South~
##  5 The British Museum                     1.33e6 4.10e6 Free      I       London
##  6 Tate Modern                            1.16e6 3.88e6 Free      I       London
##  7 Somerset House                         9.85e5 2.35e6 Free      M       London
##  8 Science Museum                         9.56e5 2.33e6 Free      I       London
##  9 Jeskyns Community Woodland             8.79e5 4.47e5 Free      O       South~
## 10 V&A South Kensington                   8.58e5 2.37e6 Free      I       London
\end{verbatim}

\subsection{Question 4}\label{question-4}

\emph{What is the admission charge for the
\texttt{"National\ Museum\ of\ Scotland"}?}

\begin{Shaded}
\begin{Highlighting}[]
\NormalTok{visitors }\SpecialCharTok{\%\textgreater{}\%} \FunctionTok{filter}\NormalTok{(attraction }\SpecialCharTok{==} \StringTok{"National Museum of Scotland"}\NormalTok{)}
\end{Highlighting}
\end{Shaded}

\begin{verbatim}
## # A tibble: 1 x 6
##   attraction                  n_2021  n_2022 admission setting region  
##   <chr>                        <dbl>   <dbl> <chr>     <chr>   <chr>   
## 1 National Museum of Scotland 660741 1973751 Free      I       Scotland
\end{verbatim}

\subsubsection{Exercise d.}\label{exercise-d.}

\emph{Which attraction had exactly 565,772 visitors in 2022?}

\begin{Shaded}
\begin{Highlighting}[]
\NormalTok{visitors }\SpecialCharTok{\%\textgreater{}\%} \FunctionTok{filter}\NormalTok{(n\_2022 }\SpecialCharTok{==} \DecValTok{565772}\NormalTok{)}
\end{Highlighting}
\end{Shaded}

\begin{verbatim}
## # A tibble: 1 x 6
##   attraction                        n_2021 n_2022 admission setting region    
##   <chr>                              <dbl>  <dbl> <chr>     <chr>   <chr>     
## 1 Knowsley Safari and Knowsley Hall     NA 565772 Members   M       North West
\end{verbatim}

\subsubsection{Exercise e.}\label{exercise-e.}

\emph{How many attraction had more than 1 million visitors in 2022?}

\begin{Shaded}
\begin{Highlighting}[]
\NormalTok{visitors }\SpecialCharTok{\%\textgreater{}\%} 
  \FunctionTok{filter}\NormalTok{(n\_2022 }\SpecialCharTok{\textgreater{}} \DecValTok{1000000}\NormalTok{) }\SpecialCharTok{\%\textgreater{}\%}
  \FunctionTok{count}\NormalTok{()}
\end{Highlighting}
\end{Shaded}

\begin{verbatim}
## # A tibble: 1 x 1
##       n
##   <int>
## 1    22
\end{verbatim}

\subsection{Question 5}\label{question-5}

\emph{How many \texttt{"O"}utside attractions are there in the
\texttt{"Yorkshire\ and\ the\ Humber"} region that gives
\texttt{"Members"} free admission, which had more than 100,000 visitors
in 2022?}

\begin{Shaded}
\begin{Highlighting}[]
\NormalTok{visitors }\SpecialCharTok{\%\textgreater{}\%}
  \FunctionTok{filter}\NormalTok{(}
\NormalTok{    setting }\SpecialCharTok{==} \StringTok{"O"}\NormalTok{,}
\NormalTok{    region }\SpecialCharTok{==} \StringTok{"Yorkshire and the Humber"}\NormalTok{,}
\NormalTok{    admission }\SpecialCharTok{==} \StringTok{"Members"}\NormalTok{,}
\NormalTok{    n\_2022 }\SpecialCharTok{\textgreater{}=} \DecValTok{100000}
\NormalTok{    ) }\SpecialCharTok{\%\textgreater{}\%}
  \FunctionTok{count}\NormalTok{()}
\end{Highlighting}
\end{Shaded}

\begin{verbatim}
## # A tibble: 1 x 1
##       n
##   <int>
## 1     3
\end{verbatim}

\subsubsection{Exercise f.}\label{exercise-f.}

\emph{How many attractions had between 50,000 and 100,000 visitors in
2022?}

\begin{Shaded}
\begin{Highlighting}[]
\NormalTok{visitors }\SpecialCharTok{\%\textgreater{}\%} 
  \FunctionTok{filter}\NormalTok{(}
\NormalTok{    n\_2022 }\SpecialCharTok{\textgreater{}} \DecValTok{50000}\NormalTok{,}
\NormalTok{    n\_2022 }\SpecialCharTok{\textless{}}\DecValTok{100000}\NormalTok{ ) }\SpecialCharTok{\%\textgreater{}\%}
  \FunctionTok{count}\NormalTok{()}
\end{Highlighting}
\end{Shaded}

\begin{verbatim}
## # A tibble: 1 x 1
##       n
##   <int>
## 1    50
\end{verbatim}

\subsubsection{Exercise g.}\label{exercise-g.}

\emph{How many regions have more than 50 tourist attraction in the data
set? (Hint: You will need to tabulate the data before filtering.)}

\begin{Shaded}
\begin{Highlighting}[]
\NormalTok{visitors }\SpecialCharTok{\%\textgreater{}\%} 
  \FunctionTok{count}\NormalTok{(region) }\SpecialCharTok{\%\textgreater{}\%}
  \FunctionTok{filter}\NormalTok{(n }\SpecialCharTok{\textgreater{}} \DecValTok{50}\NormalTok{)}
\end{Highlighting}
\end{Shaded}

\begin{verbatim}
## # A tibble: 1 x 2
##   region       n
##   <chr>    <int>
## 1 Scotland    97
\end{verbatim}

\section{Summarising Data}\label{summarising-data}

\subsection{Question 6}\label{question-6}

\emph{What are the mean and median visitor numbers in 2022 across all
attractions?}

\begin{Shaded}
\begin{Highlighting}[]
\NormalTok{visitors }\SpecialCharTok{\%\textgreater{}\%} 
  \FunctionTok{summarise}\NormalTok{(}
    \AttributeTok{mean\_2022 =} \FunctionTok{mean}\NormalTok{(n\_2022),}
    \AttributeTok{med\_2022 =} \FunctionTok{median}\NormalTok{(n\_2022)}
\NormalTok{  )}
\end{Highlighting}
\end{Shaded}

\begin{verbatim}
## # A tibble: 1 x 2
##   mean_2022 med_2022
##       <dbl>    <dbl>
## 1   351942.  184640.
\end{verbatim}

\subsubsection{Exercise h.}\label{exercise-h.}

\emph{Perform the same calculation for the 2021 admissions data.}

\begin{Shaded}
\begin{Highlighting}[]
\NormalTok{visitors }\SpecialCharTok{\%\textgreater{}\%} 
  \FunctionTok{summarise}\NormalTok{(}
    \AttributeTok{mean\_2021 =} \FunctionTok{mean}\NormalTok{(n\_2021, }\AttributeTok{na.rm =} \ConstantTok{TRUE}\NormalTok{),}
    \AttributeTok{median\_2021 =} \FunctionTok{median}\NormalTok{(n\_2021, }\AttributeTok{na.rm =} \ConstantTok{TRUE}\NormalTok{)}
\NormalTok{  )}
\end{Highlighting}
\end{Shaded}

\begin{verbatim}
## # A tibble: 1 x 2
##   mean_2021 median_2021
##       <dbl>       <dbl>
## 1   232431.      129829
\end{verbatim}

All values in the output are \texttt{NA}!

\subsubsection{Exercise i.}\label{exercise-i.}

\emph{What does \texttt{NA} stand for and why are you getting this as
your answer to the previous question.}

\textbf{NA is used to represent variables whose data is ``not
available''}

\subsubsection{Exercise j.}\label{exercise-j.}

\emph{Look at the help pages for the \texttt{mean()} and
\texttt{median()} commands to see what the input argument \texttt{na.rm}
does. Edit your code from exercise h so that it computes the summary
statistics where data is available.}

\textbf{Write your answer here}

\subsection{Question 7}\label{question-7}

\emph{Which setting (inside, outside or mixed) has the largest mean
visitor numbers in 2022?}

\begin{Shaded}
\begin{Highlighting}[]
\NormalTok{visitors }\SpecialCharTok{\%\textgreater{}\%} 
  \FunctionTok{group\_by}\NormalTok{(setting) }\SpecialCharTok{\%\textgreater{}\%}
  \FunctionTok{summarise}\NormalTok{(}
    \AttributeTok{mean\_2022 =} \FunctionTok{mean}\NormalTok{(n\_2022),}
    \AttributeTok{med\_2022 =} \FunctionTok{median}\NormalTok{(n\_2022)}
\NormalTok{  )}
\end{Highlighting}
\end{Shaded}

\begin{verbatim}
## # A tibble: 3 x 3
##   setting mean_2022 med_2022
##   <chr>       <dbl>    <dbl>
## 1 I         488522.  204823 
## 2 M         284800.  198374.
## 3 O         272077.  127708
\end{verbatim}

\subsubsection{Exercise k.}\label{exercise-k.}

\emph{Observe in question 6 that the mean statistics across all settings
are much larger than the corresponding median statistics. Discuss in
your group what this suggests about the data.}

\textbf{Write your answer here}

\subsection{Question 8}\label{question-8}

\emph{What is the interquartile range (the width of the middle 50\% of
data set between the lower and upper quartiles) the for each of the four
nations of the UK?}

\begin{Shaded}
\begin{Highlighting}[]
\NormalTok{visitors\_with\_nations }\OtherTok{\textless{}{-}}\NormalTok{ visitors }\SpecialCharTok{\%\textgreater{}\%}
  \FunctionTok{mutate}\NormalTok{(}
    \AttributeTok{nation =} \FunctionTok{case\_when}\NormalTok{(}
\NormalTok{      region }\SpecialCharTok{==} \StringTok{"Northern Ireland"} \SpecialCharTok{\textasciitilde{}} \StringTok{"Northern Ireland"}\NormalTok{,}
\NormalTok{      region }\SpecialCharTok{==} \StringTok{"Scotland"} \SpecialCharTok{\textasciitilde{}} \StringTok{"Scotland"}\NormalTok{,}
\NormalTok{      region }\SpecialCharTok{==} \StringTok{"Wales"} \SpecialCharTok{\textasciitilde{}} \StringTok{"Wales"}\NormalTok{,}
      \ConstantTok{TRUE} \SpecialCharTok{\textasciitilde{}} \StringTok{"England"}
\NormalTok{    )}
\NormalTok{  )}

\NormalTok{visitors\_with\_nations }\SpecialCharTok{\%\textgreater{}\%} 
  \FunctionTok{group\_by}\NormalTok{(nation) }\SpecialCharTok{\%\textgreater{}\%}
  \FunctionTok{summarise}\NormalTok{(}
    \AttributeTok{IQR\_2022 =} \FunctionTok{IQR}\NormalTok{(n\_2022)}
\NormalTok{  )}
\end{Highlighting}
\end{Shaded}

\begin{verbatim}
## # A tibble: 4 x 2
##   nation           IQR_2022
##   <chr>               <dbl>
## 1 England           350362.
## 2 Northern Ireland  311046 
## 3 Scotland          127986 
## 4 Wales             103368.
\end{verbatim}

\subsubsection{Exercise l.}\label{exercise-l.}

\emph{How many tourist attractions are there in each of the 4 nations?
From this, discuss in your group how reliable you think the
inter-quartile estimates are.}

\begin{Shaded}
\begin{Highlighting}[]
\NormalTok{visitors\_with\_nations }\SpecialCharTok{\%\textgreater{}\%}
  \FunctionTok{group\_by}\NormalTok{(nation) }\SpecialCharTok{|\textgreater{}} 
  \FunctionTok{count}\NormalTok{()}
\end{Highlighting}
\end{Shaded}

\begin{verbatim}
## # A tibble: 4 x 2
## # Groups:   nation [4]
##   nation               n
##   <chr>            <int>
## 1 England            240
## 2 Northern Ireland     9
## 3 Scotland            97
## 4 Wales                2
\end{verbatim}

\section{Challenging Exercises}\label{challenging-exercises}

\subsubsection{Exercise m.}\label{exercise-m.}

\emph{Within each of the 4 nations, what is the proportion of tourist
attractions that have free admission for all visitors?}

\begin{Shaded}
\begin{Highlighting}[]
\NormalTok{visitors\_with\_nations }\SpecialCharTok{|\textgreater{}} 
  \FunctionTok{select}\NormalTok{(nation, admission) }\SpecialCharTok{|\textgreater{}} 
  \FunctionTok{group\_by}\NormalTok{(nation) }\SpecialCharTok{|\textgreater{}}
  \FunctionTok{summarise}\NormalTok{(}
    \AttributeTok{tot\_atrcn =} \FunctionTok{n}\NormalTok{(),}
    \AttributeTok{fr\_atrcn =} \FunctionTok{sum}\NormalTok{(admission }\SpecialCharTok{==} \StringTok{"Free"}\NormalTok{),}
    \AttributeTok{ratip =}\NormalTok{ fr\_atrcn }\SpecialCharTok{*}\DecValTok{100} \SpecialCharTok{/}\NormalTok{ tot\_atrcn}
\NormalTok{    )}
\end{Highlighting}
\end{Shaded}

\begin{verbatim}
## # A tibble: 4 x 4
##   nation           tot_atrcn fr_atrcn ratip
##   <chr>                <int>    <int> <dbl>
## 1 England                240       93  38.8
## 2 Northern Ireland         9        1  11.1
## 3 Scotland                97       34  35.1
## 4 Wales                    2        0   0
\end{verbatim}

\subsubsection{Exercise n.}\label{exercise-n.}

\emph{Calculate the percentage change in visitor admissions from 2021 to
2022. Of the tourist attractions in Scotland, sort into increasing
numerical order the types of admission charges based on the mean
percentage change in visitor numbers.}

\begin{Shaded}
\begin{Highlighting}[]
\CommentTok{\# Write your code here \%\textgreater{}\%}
\end{Highlighting}
\end{Shaded}


\end{document}
